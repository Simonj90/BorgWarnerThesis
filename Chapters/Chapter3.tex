\chapter{Friction}
This chapter will describe friction in a more detailed manner. What it is and how it affects a car.

\section{What is friction?}
Friction is the force that resists one element of material sliding against another. There are several types of friction, one of them is dry friction which resists relative lateral motion of two solid surfaces in contact. Hence dry friction is the force that one must overcome to pull a box along a floor. Dry friction is the friction that's important for this work. Further on, dry friction can be divided in two, static friction and kinetic friction. Coulomb friction is a model used to approximate dry friction. It's expressed as:
\begin{equation} \label{eq:friction}
F_{f}\leq\mu F_{n}
\end{equation}
where:
\begin{itemize}
	\item $ F_{f} $ is the force of friction, it's parallel to the surface and has a direction opposed the applied net force.
	\item $ \mu $ is the coefficient of friction, different for different surfaces.
	\item $ F_{n} $ is the normal force exerted by each surface on the other.
\end{itemize}
This model provides a threshold for how much side force that can be applied before an object starts to move laterally. As long as the side force is less than or equal to the normal force multiplied by the coefficient of friction an equal amount of friction force will be generated in the opposite direction, thus preventing the object from moving. How much force that is needed to move an object along a surface is thus decided by two factors, the normal force acting on the object and the coefficient of friction between the two surfaces. When the applied side force gets larger than this threshold the object will start to move, hence leaving the static friction region and entering the kinetic friction region. The maximum side force that can be applied before an object starts moving is known as traction which is a common term when dealing with vehicles. It's simply how much longitudinal/lateral force a tire can handle before it loses the grip to the road and starts sliding.

The coefficient of friction for static friction is denoted as $ \mu_{s} $ and the one for kinetic friction as $ \mu_{k} $. Generally the kinetic coefficient of friction is lower than the static one. This means that the side force needed to make an object move is larger than the force needed to keep it sliding. ABS?!?!?!\todo{vad galler?} The friction coefficient between two materials needs to be determined empirically, it cannot be calculated. 

\section{Tire/road friction}
Tire/road friction is, as the name suggests, the friction between tire and road. Normally the tire road friction coefficient is within the range 0.1 - 1, 0.1 for bad tires on ice and 1 for good tires on dry asphalt. Rewriting equation \ref{eq:friction} gives:
\begin{equation} \label{eq:friction2}
\frac{F_{f}}{F_{n}} \leq\mu
\end{equation}
Having it on this form makes it easier to understand what a certain coefficient of friction really means for the vehicle. Having a coefficient of friction equal to 1 means that the force of friction can be as large as the normal force acting on the tire. This also mens that the force of friction for all tires can be as large as the normal force acting on the whole vehicle. Lets have an example. A car has a mass of 1 300 $ kg $. Lets assume that it's driving at a completely horizontal asphalt road with a coefficient of friction equal to 1. Since the road is horizontal the normal force acting on the vehicle can be expressed as:
\begin{equation} \label{eq:friction3}
F_{n}=mg \rightarrow F_{n} = 12766 N 
\end{equation}
Having a coefficient of friction equal to 1 means that the force friction can be equal to the normal force without the vehicle loosing grip. Thus, 12766 $ N $ of force can be used to accelerate the vehicle or 12766 $ N $ of force can be generated while cornering or braking without the vehicle loosing grip. Cornering and accelerating can be done at the same time, as long as the total amount of force the tires need to handle won't rise above 12766 $ N $.

The example above is extremely simplified, there are lots of other factors coming into play when a vehicle is accelerating of cornering but it still gives a good idea about the properties of tire/road friction. A coefficient of friction equal to 1 means that a force of friction equal to 100 $ \% $ of the normal force can be generate and a coefficient of friction equal to 0.1 means that a force of friction equal to 10 $ \% $ of the normal force can be generated.

The example also illustrates another important point. A vehicle can never accelerate faster or corner harder than 1 $ G $ (9.82 $ m/s^2 $). Unless some kind of downforce is generated. 

\section{Downforce}
Downforce is a downward force generated by the aerodynamics of a car. The idea is to increase the normal force acting on the car and by doing that more force of friction can be generated resulting in better grip. It's important to understand that the increased force of friction isn't due to a higher coefficient of friction, it's still the same. The increased grip comes purely from an increased normal force.