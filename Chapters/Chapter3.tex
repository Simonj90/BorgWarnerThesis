\chapter{To estimate friction}

\section{What is friction? (baby don't hurt me, don't hurt me no more)}
Friction is the force that resists one element of material sliding against another. There are several types of friction, one of them is dry friction which resists relative lateral motion of two solid surfaces in contact. Hence dry friction is the force that one must overcome to pull a box along a floor. Dry friction is the friction that's important for this work. Further on, dry friction can be divided in two, static friction and kinetic friction. Coulomb friction is a model used to approximate dry friction. It's expressed as:

\begin{equation}
F_{f}\leq\mu F_{n}
\end{equation}

where

\begin{itemize}
	\item $ F_{f} $ is the force of friction, it's parallel to the surface and has a direction opposed the applied net force.
	\item $ \mu $ is the coefficient of friction, different for different surfaces.
	\item $ F_{n} $ is the normal force exerted by each surface on the other.
\end{itemize}

This model provides a threshold for how much side force that can be applied before an object starts to move laterally. As long as the side force is less than or equal to the normal force multiplied by the coefficient of friction an equal amount of friction force will be generated in the opposite direction, thus preventing the object from moving. How much force that is needed to move an object along a surface is thus decided by two factors, the normal force acting on the object and the coefficient of friction between the two surfaces. When the applied side force gets larger than this threshold the object will start to move, hence leaving the static friction region and entering the kinetic friction region. The maximum side force that can be applied before an object starts moving is known as traction which is a common term when dealing with vehicles. It's simply how much longitudinal/lateral force a tire can handle before it loses the grip to the road and starts sliding.

The coefficient of friction for static friction is denoted as $ \mu_{s} $ and the one for kinetic friction as $ \mu_{k} $. Generally the kinetic coefficient of friction is lower than the static one. This means that the side force needed to make an object move is larger than the force needed to keep it sliding. ABS?!?!?! The friction coefficient between two materials needs to be determined empirically, it cannot be calculated. 

\section{Tire/road friction}
Tire/road friction is, as the name suggests, the friction between tire and road. Normally the tire road friction coefficient is within the range $ 0.1 - 1 $, $ 0.1 $ for bad tires on ice and $ 1 $ for good tires on dry asphalt. 

\section{Friction estimation approach}

The procedure on how the friction is to be estimated will be described later in this paper. This section is presented merely to get an understanding of the basic ideas when approaching the friction estimation problem.

First of all, the different forces acting on the vehicle will have to be calculated. This will be done by measuring or approximating signals such as wheel speeds, yaw rate and accelerations and using them in a vehicle model. The fact that all forces acting on the car (neglecting wind drag) comes from the ground and the tires, the forces from the acceleration and yaw rate of the car has to come from tire forces. 

The other step is to calculate the force generated by the tire through a tire model. These models often depend on the tire stiffness, slips, normal force and the tire/road friction coefficient. The goal then becomes choosing the right friction coefficient so that the forces from the vehicle model and the tire model align. This can for example be done by using some kind of least square fitting method. 

In a theoretical world, where the vehicle and tire models work perfect, the friction estimation becomes quite easy and can be obtained with good certainty at quick speeds. Unfortunately there exists measurement and process noise, and also that the models are shown to be insufficient in certain driving situations, which complicate the work.

\section{Tire stiffness}

There are many parameters that need to be knows about the vehicle in order to estimate the friction well. One parameter that is not known (unless specified) is the stiffness of the tire. The force generated from the ground through the tires can, in the linear region, be described as a function of the tire stiffness and the slip. The lateral and longitudinal forces can be described respectively as
\begin{equation}
	F_{x} = C_{x}*\kappa
\end{equation}
\begin{equation}
	F_{y} = C_{\alpha}*\alpha
\end{equation}

The tire stiffness is therefore said to be the amount of force generated per slip, but only valid in the linear region. Thus the theoretical definition of the tire stiffness is the gradient of the force per slip around 0 slip, longitudinal tire stiffness can therefore be seen in FIGURE?!?!??!?. 

Different tires can have very different tire stiffness, which will have a large effect on the tire models. E.g. snow tires are much softer than summer tires and therefore lower stiffness. Generally, this also means that snow tires reach their maximum generated force at a larger slip then summer tires. The tire stiffness should not have to be set beforehand (meaning that the driver would have to specify if he/she changed tires) and therefore need to be estimated to accurate the friction estimation.
\section{Restrictions}

\begin{itemize}
	\item slip angle paverkar kraften i langsled
	\item Vy ar svart att fa fram
	\item langsaccelerometer finns inte alltid, berakna acc fran bakhjul brusigt
	\item vad hander nar hjulet spinner -> mycket slip, kraften okar ju inte
	\item kraften som kravs for acceleration paverkas av lutning av vag osv.
	\item 
\end{itemize}

\section{Restrictions and conditions}
There have been lots of research and model proposals related to friction estimation the last decades. The outcome of these papers usually show promising results, where the proposed solution works very well during simulations and/or testing. It is of course very desirable for the authors to deliver positive results where a good solutions to the proposed problem. These different solution that are developed can provide a lot of help to the work described in this paper and give guidance a long way. But due to the fact that most results are based on theoretical simulations and that our solution has to work in a real environment with actual sensor signals, a lot of the information provided within the field of study can be of little use or even misleading.

The problem proposed in this work is to estimate the tire/road friction for a car using an FXD, resulting in a conditions and restrictions that have to be applied throughout this work. First of all, cars with an FXD installed are exclusively front wheel driven, meaning that there are no longitudinal forces acting on the rear tires when accelerating. Therefore the velocity of the rear wheels can in most cases be used as reference speed for the whole vehicle (actually not entirely true when cornering). Through the same reasoning, the acceleration of the car can be derived from the derivative of the rear wheel speeds. Another aspect that has to be considered is that the FXD is an active limited slip differential, which means that the torque applied to the two driven wheels can be different in certain driving situations, unlike a car with a regular open differential. The most important aspect that has to be considered is the fact that the friction model has to work in a real car handled in actual driving situations. A major fact is that the sensor signals that are to be used  will come from the cars controller area network buss (CAN bus) and therefore include both measurement and process noise leading to inaccurate signals. 

The true dynamics of a car, and especially a tire, is very complex and difficult to describe accurately. The models for described the vehicles and tire behavior work well in many driving situations, but are in some not able to capture the actual detailed dynamics. Together with the inaccurate measurements mentioned earlier, this provides a great challenge on the work of finding the estimated tire/road friction. 

There are also many parameters used in different models that are very hard to know, either by sensor measurement or by estimation. Some of these parameters include the lateral velocity, slip angle, wind drag and also some static car specific parameters like mass, wheel radius, lengths from center of gravity to the two axis and the center of gravity height. When using simulations, the exact value of these parameters can be known, but in a real environment they either have to be approximated or neglected in computations. 

\begin{equation}
\mu(F_{z})=\mu_{nominal}*(\mu_{max} + (-k)\frac{F_{z} - F_{z_{0}}}{F_{z_{0}}})
\end{equation}

\section{Friction models}

\subsection{Slip-slope friction model}

One friction model that is frequently used and referred to in research papers is the so called slip-slope friction model. The models general idea is that the maximum tire/road friction available can be decided due to its dependency on the slope from the slip-force curve in the linear region. This slip-force curve has the same characteristics as the slip-friction coefficient curve seen in Figure \ref{fric_slip}. 
\begin{equation}
	\dfrac{F_{x}}{F_{z}} = k*\kappa
\end{equation}
Where $ F_{x} $ and $ F_{z} $ is the estimated longitudinal and normal force acting on a tire depending on input values. The slip-slope can be estimated with for example recursive least square fitting.


\subsection{Kalman filter and recursive least squares}
This approach is done in two steps. First the forces on each wheel are estimated with a Kalman filter. Then the tire road friction coefficient is found by fitting a tire model to the estimated tire force.