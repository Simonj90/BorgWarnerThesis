\chapter{Discussion}

\section{Future work}
More testing with different tires need to be done. Test driving have only been done with two different sets of tires in this work which of course restricts the possibilities to create a friction coefficient estimator that works for all tires. Although, it's believed that the tire selector that has been developed can easily be extended to handle all kinds of tires but this requires a lot more test data.

The estimator could also be extended with several modules to enable better estimation of the friction coefficient. Methods to estimate $ V_{y} $ (lateral velocity) and $ \alpha $ (slip angle) could greatly improve estimations of the vehicles dynamics state. Which would further on improve the reliability of vehicle and tire models.

\section{Tire model parameters}
One of the greatest challenges in this work was to model a tire correctly. Even a normal road tire, nothing fancy at all, is an extremely complex structure to model mathematically. The most accurate tire model of today is Pacejka's Magic Formula \cite{pacejka}. The reason for the accuracy is simply that it's a semi empirical model that's actually not connected to a tires physical properties at all. It's (extremely) simply put just a customizable polynomial that can be adjusted to fit a tires force per slip ratio curve very well. This is really good in a laboratory environment for example where one specific tire needs to be modeled for a series of tests. It's also not very intuitive to change the parameters to change the model behavior. Instead of a couple of simple inputs like tire stiffness and normal force the formula requires several parameters which are hard or even impossible to obtain. Needless to say, for a real time application in a vehicle there are better options.

The tire model created by Ola at BorgWarner AB that was used was much simpler but still not as simple as the Dugoff or Brush models.  


\section{Losses}
Losses are generally hard to model correctly.

The formula for losses in the drive line where the gear ration has an impact on the efficiency certainly is interesting. It was found on a private persons own web site. The person in question is named Steven Mason and has worked at the rotational machinery and controls laboratory at University of Virginia. Via e-mail contact it was learned that the formula came from lecture notes of T.C. Scott's course in Automotive Engineering and that he hadn't done any extensive tests of it. The formula works really well and since it did it was chosen to use even though the source isn't the best. Before the formula was found it was already shown that higher gear ratio appeared to lower the efficiency in the drive line which also strengthened the belief in the formula. 

\section{Slip ratio}
The main problem regarding the slip ratio calculations is the approximation of the vehicles velocity. What's needed for a good slip ratio calculation is the wheels velocity relative the road in the exact same direction as the wheel in pointing. Thus, calculating the slip ratio based on the undriven wheel velocity only works good when driving in a straight line. Turning will make it much harder to calculate the slip ratio. To calculate the velocity in the pointing direction is hard without for example an optical sensor that actually "reads" the road in real time. The biggest improvement would be to consider the lateral speed of the vehicle as well. The estimate lateral speed was outside the scope of this work but there already exist methods to do so. By combining longitudinal and lateral speed, yaw rate and the wheel angle the wheels actual velocity relative the ground could be calculated, enabling a much better slip ratio estimation.


\section{Weight transfer}
Weight transfer should be effected by the curb stiffness. 