\chapter{Method}

\section{Vehicle forces}

First of all, the work done in this paper focuses on estimating the friction only when positive forces are acting on the vehicle. The slip ratios of the front wheels are calculated by using the reference velocity of the whole car, which is based on the velocities of the rear wheels. During braking, the rear wheels will also produce negative forces, which means that the slip ratios of the front wheels will be incorrect. 

The forces that are acting on the car during an acceleration primarily comes from the driving forces generated by the front wheels. Besides this, there are negative forces acting on the car, such as wind drag, rolling resistance, driveline and losses due to steering. 

\subsection{Forces from vehicle acceleration}

The simplest way of representing the positive forces acting on the vehicle is to use the acceleration and mass and apply Newtons second law of motion

\begin{equation}
	F = a*m
\end{equation}

If a vehicle is not equipped with an longitudinal accelerometer, the acceleration of the vehicle can be derived by derivation of the rear wheel velocities. Through the same reasoning as earlier, this last method will not produce correct results during braking. 

The resulting acceleration is merely the net force on the vehicle. The total amount of forces that is generated from the tires and up to the body also includes all of the losses acting during acceleration. 

A problematic situation occurs when the vehicle is accelerating either uo or down a slope. During an uphill acceleration, the actual forces acting on the car will be higher than the forces derived from the acceleration. When driving downwards, the acceleration will instead be too high compared to the true forces acting. 

\subsection{Forces from produced torque}

Another way of calculating the affecting forces of a vehicle is to derive the actual torque that is applied to the shafts connected to the driving wheels. The positive effect of using produced torque is that it is independent of the roads gradient. The torque applied to the shaft is the same as the torque used by the wheels, regardless of how the gravitational pull is acting in the vehicle. 

It is very uncommon for a commercially available vehicle to have a sensor which measures the driving shaft torque, but it can be derived using the engine torque measurements and the current gear ratio. The gear ratio of the vehicle is the speed exchange from the engine to the wheels. Thus

\begin{equation}
	\label{eq:GR}
	Gear Ratio = \frac{RPM_{engine}}{RPM_{wheel}}
\end{equation}
	
	
The combined torque to the two driving shafts can thereafter be calculated by

\begin{equation}
	\label{eq:tshaft}
	T_{shafts} = T_{engine}*Gear Ratio
\end{equation}

This torque will be split evenly on the two wheels, assuming that the FXD is inactive. By combining equations $ \ref{eq:GR} $ and $ \ref{eq:tshaft} $, it is also seen that the power generated by the engine and the power outputted to the shafts are equal.

\begin{equation}
	P = T_{shafts}*RPM_{wheel} = T_{engine}*RPM_{engine}
\end{equation}

If calculated correctly, the torque, $ Nm $, can easily be converted into the force, $ N $, that is generated to the tires. This force is obtained by derivation of the wheels radius

\begin{equation}
F_{tire} = \frac{T_{shaft}}{r_{wheel}}
\end{equation}