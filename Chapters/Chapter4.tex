\chapter{Method}

\section{Vehicle forces}

A limitation set for this work is that the friction estimation is only possible when positive forces are acting on the vehicle. The slip ratios of the front wheels are made from calculations based on the velocities of the rear wheels. During braking, the rear wheels will also produce negative slip, and thus negative forces, which means that the slip ratios of the front wheels will be incorrect. 

The total amount of force acting on a car during an acceleration primarily comes from the driving forces generated from the ground up through the tires. Besides this, there are also negative forces acting on the car, such as wind drag, rolling resistance, losses in the driveline and due to steering. 

\subsection{Forces from vehicle acceleration}

The simplest way of representing the positive forces acting on the vehicle is to use the acceleration and mass and apply Newtons second law of motion
\begin{equation}
	F = m \cdot a
\end{equation}
If a vehicle is not equipped with an longitudinal accelerometer, the acceleration of the vehicle can be derived by derivation of an undriven, i.e. the rear wheel velocities. Through the same reasoning as earlier, this last method will not produce correct results during braking, due to the negative slip on the rear wheels. 

This measured or derived acceleration is merely the resulting net force acting on the vehicle. The total amount of forces that is generated from the tires and up to the body have to include the losses that exist during acceleration. The resulting force that is generated therefore becomes 
\begin{equation}
F_{tires} = a*m + F_{steering loss} + F_{wind drag} + F_{rolling resistance}
\end{equation}
A problematic situation for this method occurs when the vehicle is accelerating on a gradient road. During an uphill acceleration, the actual forces acting on the car will be higher than the forces derived from the acceleration. When driving downhill, the acceleration will instead be  higher compared to true forces acting on the tires. 

\subsection{Forces from produced torque}

Another way of calculating the affecting forces of a vehicle is to derive the actual torque that is applied to the shafts connected to the driving wheels. The positive affect of using the produced torque, instead of the acceleration of the vehicle, is its independence of the roads gradient and losses such wind drag and steering losses. The torque that is applied to the shafts will always be equal to the torque out on the wheels, regardless of how the gravitational pull and other losses are acting. Each wheel connected to a driving shaft has its own moment of inertia which needs to be accelerated. The amount of torque needed to accelerate the wheel depends on its moment of inertia and the angular acceleration, where the moment of inertia is the sum of its mass multiplied by its radi squared
\begin{equation}
I = \sum m \cdot r^2
\end{equation}
\begin{equation}
	T = a \cdot I
\end{equation}
The resulting torque will generate a force between the tire and the ground.

In order to use this method for calculations of the vehicle forces, good knowledge of the momentary shaft torque is required. It is unfortunately very uncommon today for a commercially available vehicle to have such a sensor mounted, but the value can be derived from the engine torque and the current gear ratio. The gear ratio of the vehicle is the speed exchange between the engine and the wheels
\begin{equation}
	\label{eq:GR}
	Gear Ratio = \frac{RPM_{engine}}{RPM_{wheel}}
\end{equation}
The combined torque to the two driving shafts can thereafter be calculated by
\begin{equation}
	\label{eq:tshaft}
	T_{shafts} = T_{engine}*Gear Ratio
\end{equation}
This torque will be split evenly between the two shafts, assuming an open differential, e.i. the FXD is inactive. By combining equations $ \ref{eq:GR} $ and $ \ref{eq:tshaft} $, it is also seen that the power generated by the engine and the power outputted to the shafts are equal.
\begin{equation}
	P = T_{shafts}*RPM_{wheel} = T_{engine}*RPM_{engine}
\end{equation}
If calculated correctly, a torque, $ Nm $, can easily be converted into a force, $ N $, which is generated to the edge of the tires. This force is obtained by derivation by the radi of the wheel
\begin{equation}
	F_{tire} = \frac{T_{shaft}}{r_{wheel}}
\end{equation}
A restriction that has to be thought of is that the amount of force that can be generated between the tire and the road is limited by the tire/road friction, $ \mu $. Any additional torque applied to the shaft will merely serve as unwanted acceleration of the wheel, i.e. spinning.


\section{Tire forces}

slip angle not known, limitation, only longitudinal

slip, mue, normal force!

The main dynamic parameters that affect the frictional force is the slip ratio, which comes from the difference in the wheels angular and  directional velocity, the normal force, which simply depends on the weight loaded on a tire, and also the tire/road friction coefficient, $ \mu $
\begin{equation}
	F_{tire} = f(\kappa, Fz, \mu)
\end{equation}
Where $ f $ is a function depending on these three parameters. The slip ratio and the normal force can be estimated by derivation of sensors signals. As mentioned earlier, the only way to generate a positive force to a vehicle is by generating a frictional force between the tire and the road. This means that the force derived from the vehicle model is equal to the forces from the two driving wheels tire model froce.
