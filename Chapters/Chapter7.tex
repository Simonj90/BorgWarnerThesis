\chapter{Conclusion}

\section{..}
The work done in this report has focused as much as possible to have a solution that works in real live situations, rather than having a model that works perfectly during simulated testing. 

\subsection{Tire model parameters}
One of the larger insights acquire was that the difference between tire sets had a much bigger role than what was expected. It was first believed that most tires had the force peak at around similar slip ratio values, but after testing this was found to be untrue. This was the reason that the different sets of tire model parameters was decided to be used. However, if this friction approach is to be used for a real friction estimator, more testing with different tires need to be done. Test driving have only been performed with two different sets of tires in this work which of course restricts the possibilities to create a friction coefficient estimator that works for all tires. Although, it's believed that the tire selector that has been developed can easily be extended to handle all kinds of tires but this requires a lot more test data.

The estimator could also be extended with several modules to enable better estimation of the friction coefficient. Methods to estimate $ V_{y} $ (lateral velocity) and $ \alpha $ (slip angle) could greatly improve estimations of the vehicles dynamics state. Which would further on improve the reliability of vehicle and tire models.

\todo{Lista vilka signaler vi faktiskt behöver!}
