%% ----------------------------------------------------------------
%% Thesis.tex -- MAIN FILE (the one that you compile with LaTeX)
%% ---------------------------------------------------------------- 

% Set up the document
\documentclass[a4paper, 11pt, oneside]{Thesis}  % Use the "Thesis" style, based on the ECS Thesis style by Steve Gunn
\graphicspath{Figures/}  % Location of the graphics files (set up for graphics to be in PDF format)

% Include any extra LaTeX packages required
\usepackage[square, numbers, comma, sort&compress]{natbib}  % Use the "Natbib" style for the references in the Bibliography
\usepackage{verbatim}  % Needed for the "comment" environment to make LaTeX comments
\usepackage{vector}  % Allows "\bvec{}" and "\buvec{}" for "blackboard" style bold vectors in maths
\usepackage{pgf}
\usepackage{todonotes}
\usepackage{amsmath}
\hypersetup{urlcolor=blue, colorlinks=true}  % Colours hyperlinks in blue, but this can be distracting if there are many links.

%% ----------------------------------------------------------------
\begin{document}
\frontmatter      % Begin Roman style (i, ii, iii, iv...) page numbering

% Set up the Title Page
\title  {Tire/road friction estimation for front wheel driven vehicle}
\authors  {\texorpdfstring
            {\href{your web site or email address}{Simon Johansson \linebreak Viking Persson}}
            {Author Name}
            }
\addresses  {\groupname\\\deptname\\\univname}  % Do not change this here, instead these must be set in the "Thesis.cls" file, please look through it instead
\date       {\today}
\subject    {}
\keywords   {}

\maketitle
%% ----------------------------------------------------------------

\setstretch{1.3}  % It is better to have smaller font and larger line spacing than the other way round

% Define the page headers using the FancyHdr package and set up for one-sided printing
\fancyhead{}  % Clears all page headers and footers
\rhead{\thepage}  % Sets the right side header to show the page number
\lhead{}  % Clears the left side page header

\pagestyle{fancy}  % Finally, use the "fancy" page style to implement the FancyHdr headers

%% ----------------------------------------------------------------

% The Abstract Page
\addtotoc{Abstract}  % Add the "Abstract" page entry to the Contents
\abstract{
\addtocontents{toc}{\vspace{1em}}  % Add a gap in the Contents, for aesthetics

Vehicles of today are equipped with several driving enhancing systems. The Electronic Stability Program (ESP) controls the brakes of the vehicle to prevent undesirable vehicle behavior. The Anti-lock Braking System (ABS) prevents the wheels to lock up while braking hard. Many vehicles are also equipped with advanced All Wheel Drive (AWD) systems or Limited Slip Differentials (LSD) allowing for the drive torque to be almost freely distributed among the wheels. Knowing the coefficient of friction to the road is extremely beneficial for all of these systems, especially for the AWD and LSD systems to be able to optimize the control algorithms.

In this master thesis a method for estimating the tire/road friction coefficient will be developed. Focus will be put on Front Wheel Driven (FWD) vehicles equipped with an electronic Limited Slip Differential (eLSD). The eLSD in question is a newly launched product by BorgWarner AB called FXD (Front Cross Differential). This is an eLSD based on their well known fifth generation electro hydraulic clutch. Today it's controlled by a complex control algorithm to be able to handle several driving situations. It's desirable to know the tire/road friction coefficient to improve the control algorithm further. This is especially important when estimating the torque transfer through the differential.

}

\clearpage  % Abstract ended, start a new page
%% ----------------------------------------------------------------

\setstretch{1.3}  % Reset the line-spacing to 1.3 for body text (if it has changed)

% The Acknowledgements page, for thanking everyone
\acknowledgements{
\addtocontents{toc}{\vspace{1em}}  % Add a gap in the Contents, for aesthetics

We would like to thank the following persons.

\begin{tabular}{ l  l }
	& \\
	Ted Brink &  \parbox[t]{9cm}{\textit{Supervisor and mentor at BorgWarner TorqTransfer Systems AB in Landskrona}} \\
	& \\ 
	Ola Nockhammar & \parbox[t]{9cm}{\textit{Supervisor and mentor at BorgWarner TorqTransfer Systems AB in Landskrona}} \\ 
	& \\
	Tore H\"agglund & \parbox[t]{9cm}{\textit{Supervisor and mentor at Department of Automatic Control, Faculty of Engineering, Lund University}} \\ 
	& \\
	Anders Robertsson & \parbox[t]{9cm}{\textit{Examiner at Department of Automatic Control, Faculty of Engineering, Lund University}} \\
	& \\ 
\end{tabular}

We would also like to thank BorgWarner TorqTransfer Systems AB in Landskrona with employees for lending us necessary equipment and aiding us with invaluable information to help us in our work.

}
\clearpage  % End of the Acknowledgements
%% ----------------------------------------------------------------

\pagestyle{fancy}  %The page style headers have been "empty" all this time, now use the "fancy" headers as defined before to bring them back


%% ----------------------------------------------------------------
%\lhead{\emph{Contents}}  % Set the left side page header to "Contents"
\tableofcontents  % Write out the Table of Contents

%% ----------------------------------------------------------------
\setstretch{1.5}  % Set the line spacing to 1.5, this makes the following tables easier to read
\clearpage  % Start a new page
%\lhead{\emph{Abbreviations}}  % Set the left side page header to "Abbreviations"
\listofsymbols{ll}  % Include a list of Abbreviations (a table of two columns)
{
% \textbf{Acronym} & \textbf{W}hat (it) \textbf{S}tands \textbf{F}or \\
\textbf{ABS} & \textbf{A}nti-lock \textbf{B}rake \textbf{S}ystem \\
\textbf{AWD} & \textbf{A}ll \textbf{W}heel \textbf{D}riven \\
\textbf{CAN} & \textbf{C}ontroller \textbf{A}rea \textbf{N}etwork \\
\textbf{CoG} & \textbf{C}enter \textbf{o}f \textbf{G}ravity \\
\textbf{eLSD} & \textbf{e}lectronic \textbf{L}imited \textbf{S}slip \textbf{D}ifferential \\
\textbf{ESP} & \textbf{E}lectronic \textbf{S}tability \textbf{P}rogram \\
\textbf{FWD} & \textbf{F}ront \textbf{W}heel \textbf{D}riven \\
\textbf{FXD} & \textbf{F}ront \textit{Cross} \textbf{D}ifferential \\
\textbf{LSD} & \textbf{L}imited \textbf{S}lip \textbf{D}ifferential \\
\textbf{RLS} & \textbf{R}ecursive \textbf{L}east \textbf{S}quare \\
\textbf{RWD} & \textbf{R}ear \textbf{W}heel \textbf{D}riven \\


}


%% ----------------------------------------------------------------
\clearpage  %Start a new page
%\lhead{\emph{Symbols}}  % Set the left side page header to "Symbols"
\listofnomenclature{lll}  % Include a list of Symbols (a three column table)
{
% symbol & name & unit \\
$ \alpha $ & slip angle & rad \\
$ \beta $ & factor for lateral acceleration compensation & - \\
$ \delta $ & steering angle & rad \\
$ \eta $ & efficiency factor & - \\
$ \kappa $ & slip ratio & - \\
$ \lambda $ & forgetting factor, RLS & - \\
$ \xi $ & tire parameter specific to the BW tire model & - \\
$ \tau $ & factor for amount of used friction & - \\
$ \mu $ & coefficient of friction & - \\
$ \mu_{0} $ & normalized force ($ \frac{F_{x}}{F_{z}} $) & - \\
$ \dot \psi $ & yaw rate & rads$ ^{-1} $ \\
$ \omega $ & angular velocity & rads$ ^{-1} $ \\
& & \\ % Gap to separate the Roman symbols from the Greek          
$ C_{x} $ & longitudinal tire stiffness & - \\
$ C_{y} $ & lateral tire stiffness & - \\
$ F_{rr} $ & rolling resistance & N \\
$ F_{x} $ & longitudinal force & N \\
$ F_{y} $ & lateral force & N \\
$ F_{z} $ & normal force & N \\
$ l_{f} $ & distance between CoG and front axle & m \\
$ l_{r} $ & distance between CoG and rear axle & m \\
$ R_{e} $ & effective rolling radius & m \\
$ T_{w} $ & track width & m \\
$ V_{x} $ & longitudinal vehicle speed & ms$ ^{-1} $ \\
$ V_{y} $ & lateral vehicle speed & ms$ ^{-1} $ \\
}

%% ----------------------------------------------------------------
% End of the pre-able, contents and lists of things
% Begin the Dedication page

\setstretch{1.3}  % Return the line spacing back to 1.3

\pagestyle{empty}  % Page style needs to be empty for this page

\addtocontents{toc}{\vspace{2em}}  % Add a gap in the Contents, for aesthetics


%% ----------------------------------------------------------------
\mainmatter	  % Begin normal, numeric (1,2,3...) page numbering
\pagestyle{fancy}  % Return the page headers back to the "fancy" style

% Include the chapters of the thesis, as separate files
% Just uncomment the lines as you write the chapters

\chapter{Introduction}

\section{Background}
During the spring of 2013 BorgWarner AB went into production with their new product called FXD. This is a brand new eLSD for front wheel driven cars which uses some of the technology from their well known four wheel drive systems. Being able to control the applied torque on each wheel results in increased traction, better cornering performance and improved safety.

The torque control is done by a complex algorithm that uses several signals from the car. It has been shown that the tuning of this algorithm is different depending on tire stiffness and road condition. It is therefore desirable to extend the algorithm to be able to estimate tire stiffness and the tire/road friction coefficient.

\section{Project goal}
The goal of the project is estimate tire stiffness and the tire/road friction coefficient. The system needs to be fast and very robust to be trusted in all conditions. Another goal is to make an estimator that relies on signals easily available on CAN bus of the vehicle. It's supposed to work in a normal car and not only be restricted to testing in cars equipped with advanced technology to measure signals that's not normally available. 

\section{Volkswagen Golf GTi Mk7}
The latest Golf GTi from Volkswagen is equipped with the FXD. All driving data used in this work have been collected with this kind of car. Borg Warner has one of these cars in Landskrona that has been driven to collect data. The car at Borg Warner is equipped with Volkswagens DSG transmission which is an automatic transmission with the options to shift gear manually with paddles at the steering wheel. At all time in the report when a car is referred to it's this Golf GTi, if nothing else is explicitly stated. % Introduction

\chapter{Theory}

Before describing the work on friction estimates that has been done, it is important to have some knowledge within vehicle dynamics, including basic knowledge on how tires and also different differentials work. The following chapter tries to describe this so that the reader has the right basic knowledge needed.

\section{Vehicle dynamics}

When looking at vehicle dynamics, there are many variables that are interesting to know. Some of these include the vehicles yaw rate, and the velocity and force generated in both lateral and longitudinal direction. The characteristics of a whole car is very complex and these parameters are therefore impossible to calculate exactly. There exists many different car models that try to describe the characteristics of a car as adequate as possible.If these calculations are supposed to be done on a cars computer, the model has to be simple enough to not reach the cars limited computing capacity.

\subsection{The bicycle model}

The bicycle model is a rather simple model that can be used to describe vehicle dynamics when turning, e.i. when we have a yaw rate and lateral forces that are affecting the vehicle. The models major simplifications are that the mass of the car is seen as one center of gravity point and that the two front wheels and the two rear wheels are combined into one wheel respectively as can be seen in Figure \ref{bicycle_model}. These simplifications mean that there is no difference in forces on the two sides, e.i. there will be no roll effect to the outside wheel when turning in a corner. There is also an assumption made that there will be no pitch effect on the car, which means that there is no suspension system that is effecting the car. The model also assumes that there is no driving torque generated to the wheels, and therefore only lateral forces on the vehicle.

For most cornering situations, these assumptions work fine, and the model gives a good idea on how different parameters are affected. Despite this, one has to bare in mind that these assumptions could result in rather large errors during certain driving situations.

\begin{figure}[h]
	\centering
	\includegraphics[width=0.8\textwidth]{Pictures/bicycle_model}
	\caption {Bicycle Model. \cite{fordonsdynamik99}}
	\label{bicycle_model}
\end{figure}

The total lateral force acting on the car will depend on the forces that the front and rear tires contribute with combined. By using Newtons second law of physics, $ F_{y} = m*a_{y} $, the total lateral force acting on the car will be
\begin{equation} \label{eq:lateral_force}
	F_{y} = m*a_{y} = F_{yR} + cos(\delta)F_{yF} + sin(\delta)F_{xF} 
\end{equation}
where $ m $ denotes the vehicle mass, $ \delta $ is the front wheel steering angle. The acceleration in the center of gravity can be described as 
\begin{equation} \label{eq:lateral_acceleration}
	a_{y} = \dot v_{y} + \dfrac{F_{c}}{m}
\end{equation}
where $ \dot v_{y} $ is the actual change of velocity in lateral direction and the centrifugal force in the center of gravity, $ F_{c} $, depends on the yaw rate
\begin{equation} \label{eq:yaw_rate}
	\dot \psi = \dfrac{v_{x}}{R}
\end{equation}
\begin{equation} \label{eq:centrifugal_force}
	F_{c} = \dfrac{m*v^2_{x}}{R} = mv_{x}*\dot \psi
\end{equation}
Where $ R $ is the radius of the turn and $ v_{x} $ the velocity in the direction that the car is pointing. By combining Equations \ref{eq:lateral_acceleration} and \ref{eq:centrifugal_force}, the acceleration can described as
\begin{equation} \label{eq:lateral_acceleration_2}
	a_{y} = \dot v_{y} + v_{x}*\dot \psi
\end{equation}
When combining Equations \ref{eq:lateral_acceleration_2} and \ref{eq:lateral_force} the three different lateral force components that are effecting the vehicle can be described as
\begin{equation} \label{lateral_forces_2}
	F_{yR} + cos(\delta)F_{yF} + sin(\delta)F_{xF} = m*(\dot v_{y} + v_{x}\dot \psi)
\end{equation}
If taken one step further, the lateral force components can describe the torque created around the z-axis in the center of gravity 
\begin{equation} \label{lateral_torque}
	M_{z} = I_{z}*\dot \psi = a*(cos(\delta)F_{yF} + sin(\delta)F_{xF}) + b*F_{yR}
\end{equation}
where $ a $ and $ b $ are the lever lengths from the center of gravity to the front respective the rear axle. 

When looking at the bicycle model,the assumption is that the longitudinal force is negligible. This means, as mentioned earlier, that the lateral force will almost only depend on the tire slip angle created by the front wheel steering angle and the angle to the direction the front wheel is heading towards. When assuming this model with only two wheels, the slip angles for the front respective rear tires will be \cite{fordonsdynamik99}
\begin{equation} \label{eq:wheel_slip_front}
	\alpha _{F} = -arctan(\dfrac{v_{y} + \dot \psi *a}{v_{x}}) + \delta
\end{equation}
\begin{equation} \label{eq:wheel_slip_rear}
\alpha _{R} = -arctan(\dfrac{v_{y} - \dot \psi *b}{v_{x}})
\end{equation}
A number of interpretations can be made from these formulas. One is that the steering angle only directly influences the slip angles of the front wheels. Another is that if the numerator in \ref{eq:wheel_slip_front} goes to zero, which means that the lateral velocity plus the yaw rate is zero, the slip angle of the front wheel is equal to the front wheel steering angle. This means that the vehicle is gliding straight forward regardless of the steering angle. Another conclusion is that if $ \dot \psi *b $ is larger than $ v_{y} $, in \ref{eq:wheel_slip_rear}, the numerator will become less then zero and slip angle of the rear wheel will move to the other side on the x-axis. This happens when a vehicle takes a corner more aggressively, which means higher longitudinal velocity compared to the radius of the corner, leading to higher yaw rate as can be seen in \ref{eq:yaw_rate}.

The lateral forces acting on the front and rear wheels can also be described as 
\begin{equation}
	F_{yF} = C_{F}\alpha _{F}
\end{equation}
\begin{equation}
	F_{yR} = C_{R}\alpha _{R}
\end{equation}
where $ C_{F} $ and $ C_{R} $ are the lateral force coefficients for the front and rear wheels. A steering response, $ B $, can de defined from these two forces
\begin{equation}
	B = F_{yR} - F_{yF}
\end{equation}
If $ B $ is 0, we have equal amount of lateral force on the front and rear wheels, which means that there is no under- or oversteer. Theoretically, this means that with a constant steering angle, the radius of the corner will be the same for all velocities. With a $ B < 0 $, a car has more force on the front tires, leading to that less and less steering angle is needed when the velocity increases in a corner. At one point, the steering angle will have to be negative to keep the radius of the corner. This phenomena is called understeer. If $ B > 0 $ on the other hand, more force will be on the rear tires and more steering angle is needed to keep the radius of a corner when increasing the velocity. This kind of behavior is called understeer. A slight understeer is usually desired for commercially available cars because handling of the car becomes easier and the risk of skidding is reduced.


\subsection{Normal forces}

The longitudinal and lateral forces that act on the tires are dependent on the normal forces. It is therefore interesting to know how much normal force that's acting on each tire.

\subsubsection{Lateral influence}

\begin{figure}[h]
	\centering
	\includegraphics[width=0.8\textwidth]{Pictures/normal_force_lateral}
	\caption{Normal force on the car seen from the front.}
	\label{normal_force_lateral}
\end{figure}
Without any lateral acceleration, the right and left normal forces can be described as in Equation \ref{eq:normal}. 
\begin{equation} \label{eq:normal}
	F_{zL} + F_{zR} = mg
\end{equation}
With lateral acceleration, a torque will occur that changes the normal forces on the two sides. 
\begin{equation} \label{eq:normal_with_lat_acc}
	F_{zR}*T_{w} = mg*\frac{T_{w}}{2} + m*a_{y}*CGH
\end{equation}
The torque that affects the right side does in equation \ref{eq:normal_with_lat_acc} not take account for that the center of gravity height will change during driving.

When taking a corner, there will be load transfer from the inner to the outer wheel will depend on the centrifugal force and the load transfer coefficient $ \sigma_{i} $
\begin{equation}
	\Delta F_{zi} = \sigma_{i} ma_{y}
\end{equation}
\begin{equation}
	\sigma_{i} = \frac{1}{T_{w}}*( \frac{c_{\phi i}}{c_{\phi 1}+c_{\phi 2} - mgh'}h' + \frac{l-a_{i}}{l}h_{i}) 
\end{equation}
Where the \textit{i} denotes one of the axles (either front or rear), $ c $ the rolling stiffness, $ h $ the height from the ground to the axle, $ h' $ the distance from CoG to the roll axis, $ a_{1} = a $ and $ a_{2} = b $.

The final normal force on the front right wheel will be the normal force acting on the two front wheels divided by two, the unsprung mass of that one wheel, and the weight from the load transfer.
\begin{equation}
	F_{zFrontRight} = \frac{F_{zFront}}{2} + m_{UnsprungOneWheel}*g + \Delta F_{z}
\end{equation}
The problem with the calculations done above, is that the roll stiffness of both front and rear axle has to be known. The roll stiffness calculations are taken from \cite{pacejka}.



\subsubsection{Longitudinal influence}

\begin{figure}[h]
	\centering
	\includegraphics[width=0.8\textwidth]{Pictures/normal_force_longitudinal}
	\caption{Normal force on the car seen from the front.}
	\label{normal_force_longitudinal}
\end{figure}
\begin{equation} \label{eq:normal_2}
	F_{zF} + F_{zR} = mg
\end{equation}
\begin{equation} \label{eq:normal_with_long_acc}
	F_{zF}*(a+b) = mg*b - m*a_{x}*CGH
\end{equation}

For a car not in motion, the downward forces on the front and rear wheels will be equal to the gravitational force on the whole car. With longitudinal acceleration, the normal forces on the on the front respective the back will be changed, as can be seen in equation \ref{eq:normal_with_long_acc}. The normal force on the front will go down with positive acceleration. 

\section{Tire dynamics}
A gas inflated tire that is non loaded will have a radius called unloaded radius. When a tire is loaded, and therefore have a normal force acting on it from the road, it will deform against the road creating a contact area. The contact area is proportional to the load, more load gives larger contact area. The deformation of the tire will lead to a shorter distance between the center of the tire and the road, this is the loaded radius. 

A loaded tires contact area against the road can be divided into two, adhesion area and sliding area. The adhesion area is the part of the contact area that's said to adhere to the road, this means that this part haven't reached the friction limit yet, it can still handle more force without sliding. The sliding area is area that has reached the friction limit and thus has begun sliding. How this area is divided depends on a number of factors but it can basically be divided into two cases, longitudinal forces and lateral forces which will be explained further on.

\subsection{Longitudinal forces}
Rotating the tire will result in compression of the tire where the it hits the road and expansion where it leaves the road. The tire itself has a dampening effect meaning that all energy used to compress the tire won't be recovered when it expands again. This energy loss is called rolling resistance, $ R_{x} $. $ R_{x} $ is often modeled as being proportional to $ F_{z} $ with the proportionality constant $ f $ (Equation \ref{eq:rollingres}). A typical value of $ f $ is 0.015 for passenger cars \cite{rajamani}. The compression and expansion of the tire will also move the normal force acting on the tire in front of the center line (Figure \ref{rolling_resistance}) when the tire is rolling. The moved normal force will result in a third radius of the tire, the effective rolling radius. This is the radius related to the actual linear longitudinal velocity of the rolling tire, it's longer than the loaded radius but shorter than the unloaded radius.
\begin{equation}
R_{x} = fF_{z}
\label{eq:rollingres}
\end{equation}
\begin{figure}[h]
	\centering
	\includegraphics[width=0.8\textwidth]{Pictures/rolling_resistance}
	\caption{Normal force acting on the tire.}
	\label{rolling_resistance}
\end{figure}
When a longitudinal force is acting on the tire, traction or braking, the tire will have more compression/expansion and a slip will occur. The longitudinal slip, $ \kappa $ is defined as
\begin{equation}
\kappa = \dfrac{R_{e}\omega-V_{x}}{V_{x}}
\end{equation}
Where $V_{x}$ is the nominal velocity, $R_{e}$ the effective rolling radius, and $\omega$ the angular velocity of the wheel. This leads to the following slip with different angular velocities:
\begin{equation}
Braking: \omega = 0 \Rightarrow \kappa = -1
\end{equation}
\begin{equation}
Free rolling: \omega = \frac{V}{R_{e}} \Rightarrow \kappa = 0
\end{equation}
\begin{equation}
Spinning: \omega = 2\frac{V}{R_{e}} \Rightarrow \kappa = 1
\end{equation}
The longitudinal force that will be acquired depends on the normal force, $ F_{z} $ and the friction used, $ \mu(s) $.
\begin{equation}
F_{x} = F_{z}\mu(s)
\end{equation}
The used friction will increase with the slip ratio until the maximum friction is met. This is when full sliding occurs. 

\subsection{Lateral forces}

For a given slip angle one part of the contact area will adhere to the road and one part will slide against the road. The part that is sliding has reached the friction limit and the part that is adhering can still handle more force. As the slip angle increases a larger lateral force will occur forcing a bigger part of the contact area into the sliding region. At a certain slip angle the whole contact area will be in the sliding region. When this happens the lateral force for that tire has reached its maximum, turning more won't affect the vehicle anymore. This is well illustrated later on in \textit{Brush model (\ref{sec:brush})}.

Cornering stiffness
\begin{equation}
C_{y} = \frac{\delta F_{y}}{\delta}
\end{equation}
Where $\delta$ is the steering angle.

Self aligning torque
\begin{equation}
M_{z} = F_{y}t_{p}
\end{equation}
Where $ F_{y} $  is the lateral force acting on the tire and $ t_{p} $ its distance to the center of the wheel. 

The lateral force acting on the wheel depends on its slip angle
\begin{equation}
F_{y}=C_{F}\alpha
\end{equation}
This can only be used in the linear part of the tire force/slip angle relationship. 

\section{Tire models}
There are several models to describe a tire mathematically. These models can be divided into four categories, empirical models, semi-empirical models, simple physical models and complex physical  models. 

Empirical models describe tire characteristics that are acquired from measurements of the tire. To fit the curve according to measured data the parameters are assessed with methods like regression. A well-known empirical model is the Magic Formula \cite{pacejka}. This model provides good fit for $F_{x}$, $F_{y}$ and $M_{z}$ curves and have coefficients that's easy to interpret.

Models using for example the similarity method are semi-empirical which means that some calculations are replaced by known or measured data. By distorting, rescaling and multiplying the result new relationships are acquired which can describe the tire in different situations. For example one can observe that that the pure slip curves shape doesn't change much \cite{pacejka} when the tire runs on different conditions. By shifting the nominal curve these conditions can be described.

The physical models are purely analytical and aims to describe the tire with the help of its physical characteristics. A simple physical model uses simple mechanical representation and can be calculated fairly easy by hand. This often results in pretty poor accuracy but sometimes that's enough. To get better accuracy a more complex model can be set up and simulated in a computer using aids like the finite element method. 

\begin{figure}[h]
	\centering
	\includegraphics[width=0.8\textwidth]{Pictures/tire_modeling}
	\caption{Four categories of possible types of approach to develop a tire model. \cite{pacejka}}
	\label{tire_modeling}
\end{figure}

In Figure \ref{tire_modeling} some modeling characteristics and how they behave depending on category can be seen.

\subsection{Brush model}
\label{sec:brush}
The \textit{brush model} is a highly used \textit{simple physical model}. The idea is to model the tire surface as a row of elastic bristles that deflects in different directions depending on how the tire is loaded. This model is illustrated to the left in Figure \ref{brush1}.

\begin{figure}[h]
	\centering
	\includegraphics[width=0.8\textwidth]{Pictures/brush1}
	\caption{The brush tire model. \cite{pacejka}}
	\label{brush1}
\end{figure}

For pure side slip the bristles will deflect in the direction of the y-axis, this can be seen at the top right in Figure \ref{brush1}. In the same figure pure brake slip can be seen, that is when the bristles deflects in the direction of the x-axis. Finally at the bottom right of this figure combined slip is illustrated.

\begin{figure}[h]
	\centering
	\includegraphics[width=0.8\textwidth]{Pictures/brush2}
	\caption{The brush tire model. \cite{pacejka}}
	\label{brush2}
\end{figure}

In Figure \ref{brush2} it can be seen how different slip angles affects the tire. Small slip angles gives a large adhesion area (flat part) and a small sliding area (curved part). As the slip angle increases a larger number of bristles reaches their maximum deflection, hence increasing the sliding area. At a certain slip angle all bristles have reached their maximum deflection and this results in full sliding. 

\subsection{Magic formula}

\section{Differentials}

\subsection{Open differential}

With an open differential, the torque will be evenly distributed on the two wheels, but the velocities can be different. This means that the inner wheel will have a lower velocity in a corner, due tu its smaller turning radius. When one wheel has significant lower resistant then the other, its velocity will become much higher due to the same amount of torque. Two possible scenarios for this is when you climb a hill with different friction on the two tires and when you take corner fast leading to the inner wheel lifting from the ground. Both of these scenarios will lead to very high speeds on the wheel on the low $ \mu $ surface respectively the inner lifting wheel. The torque that the non-spinning wheel can transfer to the ground will be the same amount that the spinning wheel can transfer, leading to significant lower traction than what actually is available on the wheels.

A solution in these different scenarios would be to lock the differential. This would lead to that the two wheels would have the exact same velocity, and therefore not the same torque in certain situations. 

Side-gear and crown wheel angular velocities:
\begin{equation}
	-1 = \frac{\omega_{1} - \omega_{r}}{\omega_{2} - \omega_{r}}
\end{equation}
\begin{equation}
	\omega_{r} = \frac{\omega_{1} + \omega_{2}}{2}
\end{equation}

\subsection{Limited slip differential}

The goal with a limited slip differential (LSD) is to not apply more torque to a wheel then it can transfer to the ground.

\subsection{FXD}

Does not create under steering like locking torque LSD's. 

Lightweight alternative  to improve traction performance. Very little extra fuel consumption. Pre-emptive lock torque for take off. Wheel slip control. Yaw damping. 

If ABS or ESP is active, the FXD function is shutout. 

"Side pull reduction"


\subsubsection{Control algorithm}

Signals available for the control algorithm through CAN:
\begin{itemize}
	\item 4x Wheel speeds
	\item Engine torque
	\item Engine speed
	\item Accelerator position
	\item Brake Pedal Active
	\item ABS active
	\item ESP torque request, opening request (slow/fast)
	\item Yaw rate
	\item Steering wheel angle
	\item Lateral acceleration
\end{itemize}

Other signals that are used are estimated:

Vehicle speed.
Tire size difference.
Tire stiffness???
Lateral acceleration???



\subsubsection{Applying pressure}

By applying higher current to the pump, it will spin faster and thus creating larger centrifugal forces. This force will block the oil from exiting the pump, leading to a higher pressure within the pump. This pressure is also applied to the clutch which will press the clutch discs together and locking the axles together.

\subsection{How differentials affect vehicle handling} 

\section{Tire/road friction models/methods}

There have been extensive research within this field for the last fifty years and this results in many different approaches to model tire/road friction. Some of the more relevant models/methods for this work will be presented here.

\subsection{Mue}

\begin{equation}
	\mu(F_{z})=\mu_{nominal}*(\mu_{max} + (-k)\frac{F_{z} - F_{z_{0}}}{F_{z_{0}}})
\end{equation}

\subsection{Slip-based friction model}
 % Background Theory 

\chapter{To estimate friction}

\section{What is friction? (baby don't Force me, don't Force me no more)}
Friction is the force that resists one element of material sliding against another. There are several types of friction, one of them is dry friction which resists relative lateral motion of two solid surfaces in contact. Hence dry friction is the force that one must overcome to pull a box along a floor. Dry friction is the friction that's important for this work. Further on, dry friction can be divided in two, static friction and kinetic friction. Coulomb friction is a model used to approximate dry friction. It's expressed as:

\begin{equation} \label{eq:friction}
F_{f}\leq\mu F_{n}
\end{equation}

where

\begin{itemize}
	\item $ F_{f} $ is the force of friction, it's parallel to the surface and has a direction opposed the applied net force.
	\item $ \mu $ is the coefficient of friction, different for different surfaces.
	\item $ F_{n} $ is the normal force exerted by each surface on the other.
\end{itemize}

This model provides a threshold for how much side force that can be applied before an object starts to move laterally. As long as the side force is less than or equal to the normal force multiplied by the coefficient of friction an equal amount of friction force will be generated in the opposite direction, thus preventing the object from moving. How much force that is needed to move an object along a surface is thus decided by two factors, the normal force acting on the object and the coefficient of friction between the two surfaces. When the applied side force gets larger than this threshold the object will start to move, hence leaving the static friction region and entering the kinetic friction region. The maximum side force that can be applied before an object starts moving is known as traction which is a common term when dealing with vehicles. It's simply how much longitudinal/lateral force a tire can handle before it loses the grip to the road and starts sliding.

The coefficient of friction for static friction is denoted as $ \mu_{s} $ and the one for kinetic friction as $ \mu_{k} $. Generally the kinetic coefficient of friction is lower than the static one. This means that the side force needed to make an object move is larger than the force needed to keep it sliding. ABS?!?!?! The friction coefficient between two materials needs to be determined empirically, it cannot be calculated. 

\section{Tire/road friction}
Tire/road friction is, as the name suggests, the friction between tire and road. Normally the tire road friction coefficient is within the range 0.1 - 1, 0.1 for bad tires on ice and 1 for good tires on dry asphalt. Rewriting equation \ref{eq:friction} gives:

\begin{equation} \label{eq:friction2}
\frac{F_{f}}{F_{n}} \leq\mu
\end{equation}

Having it on this form makes it easier to understand what a certain coefficient of friction really means for the vehicle. Having a coefficient of friction equal to 1 means that the force of friction can be as large as the normal force acting on the tire. This also mens that the force of friction for all tires can be as large as the normal force acting on the whole vehicle. Lets have an example. A car has a mass of 1 300 $ kg $. Lets assume that it's driving at a completely horizontal asphalt road with a coefficient of friction equal to 1. Since the road is horizontal the normal force acting on the vehicle can be expressed as:

\begin{equation} \label{eq:friction3}
F_{n}=mg \rightarrow F_{n} = 12766 N 
\end{equation}

Having a coefficient of friction equal to 1 means that the force friction can be equal to the normal force without the vehicle loosing grip. Thus, 12766 $ N $ of force can be used to accelerate the vehicle or 12766 $ N $ of force can be generated while cornering or braking without the vehicle loosing grip. Cornering and accelerating can be done at the same time, as long as the total amount of force the tires need to handle won't rise above 12766 $ N $.

The example above is extremely simplified, there are lots of other factors coming into play when a vehicle is accelerating of cornering but it still gives a good idea about the properties of tire/road friction. A coefficient of friction equal to 1 means that a force of friction equal to 100 $ \% $ of the normal force can be generate and a coefficient of friction equal to 0.1 means that a force of friction equal to 10 $ \% $ of the normal force can be generated.

The example also illustrates another important point. A vehicle can never accelerate faster or corner harder than 1 $ G $ (9.82 $ m/s^2 $). Unless some kind of downforce is generated. 

\section{Downforce}
Downforce is a downward force generated by the aerodynamics of a car. The idea is to increase the normal force acting on the car and by doing that more force of friction can be generated resulting in better grip. It's important to understand that the increased force of friction isn't due to a higher coefficient of friction, it's still the same. The increased grip comes purely from an increased normal force.

\section{Friction estimation approach}

The procedure on how the friction is to be estimated will be described later. This section is presented to get an understanding of the friction estimation approach that is used in this paper.

The first step includes calculating the forces acting on the tires by using a vehicle model. This is done by using parameters such as wheel speed, yaw rate and acceleration, and the fact that all forces acting on the car (neglecting wind drag) has to come from the ground through the tires. The second step is to calculate the force generated by the tire through a tire model. Such a model often depend on the tire stiffness, tire slips, normal force and the road friction coefficient. By changing the friction coefficient appropriately, the forces from the two separate models will be the same. This can for example be achieved by using some kind of least square fitting method. 

In a theoretical world, where a vehicle and tire model work perfect, the friction estimation becomes relatively easy and can be obtained with good certainty quite fast. Unfortunately there exists measurement and process noise in a real car and also an uncertainty from the models in some driving situations, which further complicate the work.

\section{Tire stiffness}

There are many parameters that need to be knows about the vehicle in order to estimate the friction well. One parameter that is not known (unless specified) is the stiffness of the tire. The force generated from the ground through the tires can, in the linear region, be described as a function of the tire stiffness and the slip. The lateral and longitudinal forces can be described respectively as
\begin{equation}
	F_{x} = C_{x}*\kappa
\end{equation}
\begin{equation}
	F_{y} = C_{\alpha}*\alpha
\end{equation}

The tire stiffness is therefore said to be the amount of force generated per slip, but only valid in the linear region. Thus the theoretical definition of the tire stiffness is the gradient of the force per slip around 0 slip, longitudinal tire stiffness can therefore be seen in FIGURE?!?!??!?. 

Different tires can have very different tire stiffness, which will have a large effect on the tire models. E.g. snow tires are much softer than summer tires and therefore lower stiffness. Generally, this also means that snow tires reach their maximum generated force at a larger slip then summer tires. The tire stiffness should not have to be set beforehand (meaning that the driver would have to specify if he/she changed tires) and therefore need to be estimated to accurate the friction estimation.
\section{Restrictions}

\begin{itemize}
	\item slip angle paverkar kraften i langsled
	\item Vy ar svart att fa fram
	\item langsaccelerometer finns inte alltid, berakna acc fran bakhjul brusigt
	\item vad hander nar hjulet spinner -> mycket slip, kraften okar ju inte
	\item kraften som kravs for acceleration paverkas av lutning av vag osv.
	\item 
\end{itemize}

\section{Restrictions and conditions, aoeh}

\subsection{Related work}

There have been lots of research and model proposals related to friction estimation during the last decades. The outcome of these papers usually show promising results, where the proposed solution works well during simulations and/or testing. The related work can provide a lot of information and help to this paper, especially to get the general understanding of the problem nd its difficulties. But due to the fact that most results are based on theoretical simulations and that this solution has to work in a real environment with actual sensor signals, a lot of the information provided through previous work within the field of study can be of little use or sometimes even misleading.

\subsection{Using an FXD}

The problem stated in this work is to estimate the tire/road friction for a car using an FXD, resulting in a number of conditions and restrictions that have to be thought of and applied throughout this work. First of all, cars with an FXD installed are solely front wheel driven, meaning that there are no positive longitudinal forces acting on the rear tires when accelerating. The velocity of the rear wheels can therefore in most cases be used as a good approximation for vehicles reference speed. Through the same reasoning, the longitudinal acceleration of the car can be derived from the derivative of the rear wheel velocities. Another aspect that has to be considered is that the FXD is an active limited slip differential, which means that the torque applied to the two driven shafts can be different in certain driving situations, unlike a car equipped with a standard open differential. 

\subsection{Theoretical vs. Practical :;.;:;pP;.,P,}

The true dynamics of a car, and especially a tire, is very complex and therefore difficult to describe accurately with a model. At the same time, a model has to be simple enough so that the calculations are possible on an engine control unit that has limited computational power. Even though many simplified models are shown to be accurate enough to reflect reality in most driving situations, there are times when a simplified model inaccurately describes the detailed dynamics of a vehicle or a tire. 

There are also numerous models that use different parameters that are hard to measure or approximate well in reality. Some of these parameters include the lateral velocity and slip angle. It is therefore desired to have a model that doesn't rely on these parameters. There are also car specific parameters that have impact on the models, some that can even change between driving sequences. A few of these parameters include the mass of the vehicle, wheel radius, wind drag coefficient, lengths from center of gravity to the two axis, and the center of gravity height. When using simulations, the exact value of these parameters can be known, but in a real environment they either have to be approximated or neglected in computations.

Together with the inaccurate measurements and process of signals mentioned earlier, it is a great challenge on the work of finding the estimated tire/road friction.  


The most important aspect that has to be considered is the fact that the friction estimation model has to work in a real car handled in actual driving situations. A major fact is that the sensor signals that are to be used  will come from the cars controller area network buss (CAN bus) and therefore include both measurement and process noise leading to inaccurate signals. 

\section{Friction models}

\begin{equation}
\mu(F_{z})=\mu_{nominal}*(\mu_{max} + (-k)\frac{F_{z} - F_{z_{0}}}{F_{z_{0}}})
\end{equation}

\subsection{Slip-slope friction model}

One friction model that is frequently used and referred to in research papers is the so called slip-slope friction model. The models general idea is that the maximum tire/road friction available can be decided due to its dependency on the slope from the slip-force curve in the linear region. This slip-force curve has the same characteristics as the slip-friction coefficient curve seen in Figure \ref{fric_slip}. 
\begin{equation}
	\dfrac{F_{x}}{F_{z}} = k*\kappa
\end{equation}
Where $ F_{x} $ and $ F_{z} $ is the estimated longitudinal and normal force acting on a tire depending on input values. The slip-slope can be estimated with for example recursive least square fitting.


\subsection{Kalman filter and recursive least squares}
This approach is done in two steps. First the forces on each wheel are estimated with a Kalman filter. Then the tire road friction coefficient is found by fitting a tire model to the estimated tire force. % Friction Estimation

\chapter{Method}

\section{Krafter}
adsfa % Method

\chapter{Results}

\section{Said "What waht"}

\section{Winter tires}

\subsection{Fast track run}



\subsection{Two different mues}
The main goal for the work done in this report was to detect when low-$ \mu $ is present and thereafter be able to limit the torque transfer through the FXD. It is therefore essential to test the developed algorithm during a driving sequence that actually includes a changing $ \mu $. It has not been possible to test this on a single run, for example on asphalt and a skid pad. Therefore, two different runs have been merged together, where the friction coefficient changes three times.

The normalized force per slip ratio for this combined sequence can be seen in Figure \ref{slip_kraft_comb2}. Note that this figure is is merely Figure \ref{slip_kraft_ljungby} and \ref{slip_kraft_is} merged together, with some erroneous result when the friction coefficient changes come. 

\begin{figure}[h]
	\centering
	\includegraphics[width=1.0\textwidth]{Pictures/slip_kraft_comb2}
	\caption {Force per slip ratio for the combined driving sequence with both low- and high-$ \mu $.}
	\label{slip_kraft_comb2}
\end{figure} % Results

\chapter{Discussion}

\section{Future work}

test with different tires!

\section{Tire model parameters}
When the work started, we had no idea that different tires would differ so much. Very very important to know what kind of tires we have!!!!

\section{Losses}
Try to describe the losses better. Driveline losses, how good are they? % Discussion

\chapter{Conclusion}

The work done in this report has focused as much as possible on having a solution that works in real life situations, rather than having a model that works perfectly during simulated testing. 

\section{Tire model parameters}
One of the larger insights acquire was that the difference between tire sets had a much bigger impact than what was expected. It was first believed that most tires had the force peak at similar slip ratio values, but after extensive testing this was found not to be true. Therefore the tire model needed to be extended to handle both different tires and road conditions. However, test driving have only been performed with two different sets of tires which of course restricts the possibilities to create a friction coefficient estimator that works for all tires. It's believed that the tire selector that has been developed can easily be extended to handle all kinds of tires but this requires a lot more test data from test driving with different kinds of tires and road conditions.

The estimator in whole could also be extended with several modules to enable better estimation of the friction coefficient. Methods to estimate $ V_{y} $ (lateral velocity) and $ \alpha $ (slip angle) could greatly improve estimation of the vehicle states. This would further on improve the reliability of vehicle and tire models.

\section{CAN signals used}
A goal with this report was to use signals that exist on most new vehicles of today, rather than using parameters that need new sensors or are hard to approximate. The CAN signals used in this report to approximate the friction was:
\begin{itemize}
	\item wheel speed x4
	\item engine torque
	\item lateral acceleration
	\item gear
	\item steering wheel angle
	\item FXD-moment
\end{itemize}
Besides this, static parameters associated with the Golf GTI was used.  
 % Conclusion

%% ----------------------------------------------------------------
% Now begin the Appendices, including them as separate files

\addtocontents{toc}{\vspace{2em}} % Add a gap in the Contents, for aesthetics

\appendix % Cue to tell LaTeX that the following 'chapters' are Appendices

%\chapter{An Appendix}

Appendix text!	% Appendix Title

%\input{Appendices/AppendixB} % Appendix Title

%\input{Appendices/AppendixC} % Appendix Title

\addtocontents{toc}{\vspace{2em}}  % Add a gap in the Contents, for aesthetics
\backmatter

%% ----------------------------------------------------------------
\label{Bibliography}
%\lhead{\emph{Bibliography}}  % Change the left side page header to "Bibliography"
\bibliographystyle{unsrtnat}  % Use the "unsrtnat" BibTeX style for formatting the Bibliography
\bibliography{Bibliography}  % The references (bibliography) information are stored in the file named "Bibliography.bib"

\end{document}  % The End
%% ----------------------------------------------------------------